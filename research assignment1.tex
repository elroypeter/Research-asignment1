\documentclass[11pt]{article}
%Gummi|062|=)
\title{\textbf{THE DISRUPTIVE NATURE OF MOBILE PHONES DURING STAFF MEETING}}
\author{MUKASA PETER : 16/U/670 : 216000771 : CS}
\date{}
\begin{document}

\maketitle

\section{Introduction}
 There has been a massive increase in the use of personal mobile phones over the past years and there is every indication that this will continue. Currently at Buddo secondary school 90\% of staff have personal mobile phones. Recently a number of staff have complained about the use of personal mobile phones in meetings and asked what the official policy is. At present there is no official policy regarding phone use. This report examines the issue of mobile phone usage in staff meetings. For the purposes of this research document a personal mobile phone is a personally funded phone for private calls as opposed to an employer funded phone that directly relates to carrying out a particular job.

\section{Research Methods}
Using Quantitative\footnote{valuating data into figures to sustain critical analysis} research methods questionnaires were provided to investigate Buddo SS staff members� attitudes to the use of mobile phones in staff meetings. A total of 37 questionnaires were distributed and provided open ended responses for additional comments. Survey collections were collected in the staff room suggestion box located in the staff room. No personal information was collected, the survey was voluntary and anonymous and the data was tabulated as shown on the next page. 

\begin{center}
\begin{tabular}{|c|c|c|}
\hline
Catergory & Agree & Disagree \\ [0.5ex]
\hline
Not a problem & 32\% & 58\% \\ [0.5ex]
\hline
An issue & 86\% & 14\% \\ [0.5ex]
\hline
Disruptive & 72\% & 28\% \\ [0.5ex]
\hline
Permissible & 56\% & 44\% \\ [0.5ex]
\hline
Turned off & 70\% & 30\% \\ [0.5ex]
\hline
\end{tabular}
\end{center}
Analytical\footnote{Research technique where information and conclusions are drawn from given data} research it can be seen from the results in table that personal mobile phone use is considered to be a problem, however it was acknowledged that in some situations it should be permissible. 72\% of staff considered mobile phones to be highly disruptive and there was strong support for phones being turned off in meetings. Only 32\% thought that mobile phone usage in staff and team meetings was not a problem, whereas 86\% felt it was an issue. The results are consistent throughout the survey. Many of the respondents felt that in exceptional circumstances mobile phones should be allowed, eg medical, but there should be protocols regarding this.
\section{Conclusion} 
The use of mobile phones in staff meetings is clearly disruptive and they should be switched off. Most staff felt it is not necessary to receive personal phone calls in staff meetings except under certain circumstances, but permission should first be sought from the team leader, manager or chair. \section{Recommendations} It is recommended that Buddo SS develops an official policy regarding the use of mobile phones in staff meetings. The policy should recommend: � mobile phones are banned in staff meetings � mobiles phone may be used in exceptional circumstances but only with the permission of the appropriate manager or chair Finally, the policy needs to apply to all staff in the school. 


\end{document}
